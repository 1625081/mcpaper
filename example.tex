% !Mode:: "TeX:UTF-8"
%!TEX program  = xelatex

%\documentclass{cumcmthesis}
\documentclass[withoutpreface,bwprint]{cumcmthesis} %去掉封面与编号页

\usepackage{url}
\title{出于蹭区块链热度的知识产权系统}
\tihao{A}
%\baominghao{4321}
%\schoolname{XX大学}
%\membera{小米}
%\memberb{向左}
%\memberc{哈哈}
%\supervisor{老师}
%\yearinput{2017}
%\monthinput{08}
%\dayinput{22}

\begin{document}
 \maketitle
 \begin{abstract}
夜夜夜夜

\keywords{区块链\quad 知识产权\quad 去中心化}
\end{abstract}

%目录
\tableofcontents

\section{问题重述}

\section{模型假设与符号说明}
\subsection{模型假设}
\subsection{符号说明}

\section{问题分析与背景介绍}
\subsection{问题分析}
分析一下题目要求与满足这些要求的模型类别——就是基于区块链的模型
\subsection{背景介绍}
这一部分主要想介绍一下已有的系统与模型。章节名可以另起。

\section{模型}
\subsection{模型综述}

系统中的每名用户拥有唯一的id(user id)与密钥,用于记录(transaction)的对象与记录有效性认证。

记录有如下三类:
\begin{itemize}
\item{作品授权}

该类记录须经版权所有者认证,其格式为:记录类型(transcation type)+作品的mediachainid+授权人用户id+被授权人id+权限类型+版权所有者签名。其中记录类型为“作品授权”。权限类型为“仅使用”、“使用与再创作”、“使用与再授权”、“转移”等。

比较特殊的情况为创作者声明自己作品的版权。创作者应当首先将作品提交至mediachain系统,获得相应的id。被授权用户为创作者id,权限类型为“转移”,作品首先由创作者创造,并不存在真实的授权人,因此引入一特殊用户“所有人(Human)”,任何首创的作品被认为是由“所有人”将版权转移给创作者。

\item{资金流动}
该类记录须经资金持有者认证,其格式为:记录类型+交易金额+付款人用户id+收款人用户id+付款人签名+收款人签名。记录类型为“资金流动 ”。

与其他区块链基础的货币不同,知识产权系统的货币仅为方便交易的数字记录(类似于支付宝而不是比特币),因此不仅仅存在内部的货币交易,也存在充值与体现等问题。因此系统设置一特殊用户“银行(Bank)”。用户在现实世界向银行交钱或者取钱,银行提供相应的记录证明并签字。如果为充值行为即银行向充值人付款,提现即为提现人向银行付款。
\end{itemize}

在用户生产相应的记录请求完成后,向全网广播,发出记录请求(transcation request),系统根据记录请求中含有的相关内容搜索产权记录链(intellectual property chain,IPC),核实记录请求的有效性(validation)。依照记录类型的不同,分为

\begin{itemize}
	\item{作品授权}
	检验签名是否为授权人用户id的签名,其次根据IPC中的记录验证授权人是否具有授权资格。如果通过验证,则为有效记录;否则为无效记录。记录请求被驳回,并通知有权力出售该作品版权的用户。注意到一旦完成创作作品的版权就由“所有人”转移给了作者,因此此时作者是唯一的授权权利人。
	\item{资金流动}
	验证签名者是否为付款人,然后根据IPC记录验证授权人余额是否足以支付本次付款。如果通过验证,则为有效记录,否则为无效记录,驳回记录请求。注意到银行的余额应被判定为无穷。
\end{itemize}

如果被判定为有效记录(valid transcation),则进入记录缓冲区(transcation buffer)。每隔一段时间,基于一定的共识协议,由系统中的某一名用户将缓冲区内的记录作为新的区块添加至IPC末端并广播




\subsection{知识产权的声明与保护}

\subsection{知识产权的收益与分配}

\section{总结}

\section*{附件:项目建议书}
\end{document} 